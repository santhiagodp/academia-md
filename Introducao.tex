
% Default to the notebook output style

    


% Inherit from the specified cell style.




    
\documentclass{article}

    
    
    \usepackage{graphicx} % Used to insert images
    \usepackage{adjustbox} % Used to constrain images to a maximum size 
    \usepackage{color} % Allow colors to be defined
    \usepackage{enumerate} % Needed for markdown enumerations to work
    \usepackage{geometry} % Used to adjust the document margins
    \usepackage{amsmath} % Equations
    \usepackage{amssymb} % Equations
    \usepackage[mathletters]{ucs} % Extended unicode (utf-8) support
    \usepackage[utf8x]{inputenc} % Allow utf-8 characters in the tex document
    \usepackage{fancyvrb} % verbatim replacement that allows latex
    \usepackage{grffile} % extends the file name processing of package graphics 
                         % to support a larger range 
    % The hyperref package gives us a pdf with properly built
    % internal navigation ('pdf bookmarks' for the table of contents,
    % internal cross-reference links, web links for URLs, etc.)
    \usepackage{hyperref}
    \usepackage{longtable} % longtable support required by pandoc >1.10
    \usepackage{booktabs}  % table support for pandoc > 1.12.2
    

    
    
    \definecolor{orange}{cmyk}{0,0.4,0.8,0.2}
    \definecolor{darkorange}{rgb}{.71,0.21,0.01}
    \definecolor{darkgreen}{rgb}{.12,.54,.11}
    \definecolor{myteal}{rgb}{.26, .44, .56}
    \definecolor{gray}{gray}{0.45}
    \definecolor{lightgray}{gray}{.95}
    \definecolor{mediumgray}{gray}{.8}
    \definecolor{inputbackground}{rgb}{.95, .95, .85}
    \definecolor{outputbackground}{rgb}{.95, .95, .95}
    \definecolor{traceback}{rgb}{1, .95, .95}
    % ansi colors
    \definecolor{red}{rgb}{.6,0,0}
    \definecolor{green}{rgb}{0,.65,0}
    \definecolor{brown}{rgb}{0.6,0.6,0}
    \definecolor{blue}{rgb}{0,.145,.698}
    \definecolor{purple}{rgb}{.698,.145,.698}
    \definecolor{cyan}{rgb}{0,.698,.698}
    \definecolor{lightgray}{gray}{0.5}
    
    % bright ansi colors
    \definecolor{darkgray}{gray}{0.25}
    \definecolor{lightred}{rgb}{1.0,0.39,0.28}
    \definecolor{lightgreen}{rgb}{0.48,0.99,0.0}
    \definecolor{lightblue}{rgb}{0.53,0.81,0.92}
    \definecolor{lightpurple}{rgb}{0.87,0.63,0.87}
    \definecolor{lightcyan}{rgb}{0.5,1.0,0.83}
    
    % commands and environments needed by pandoc snippets
    % extracted from the output of `pandoc -s`
    \DefineVerbatimEnvironment{Highlighting}{Verbatim}{commandchars=\\\{\}}
    % Add ',fontsize=\small' for more characters per line
    \newenvironment{Shaded}{}{}
    \newcommand{\KeywordTok}[1]{\textcolor[rgb]{0.00,0.44,0.13}{\textbf{{#1}}}}
    \newcommand{\DataTypeTok}[1]{\textcolor[rgb]{0.56,0.13,0.00}{{#1}}}
    \newcommand{\DecValTok}[1]{\textcolor[rgb]{0.25,0.63,0.44}{{#1}}}
    \newcommand{\BaseNTok}[1]{\textcolor[rgb]{0.25,0.63,0.44}{{#1}}}
    \newcommand{\FloatTok}[1]{\textcolor[rgb]{0.25,0.63,0.44}{{#1}}}
    \newcommand{\CharTok}[1]{\textcolor[rgb]{0.25,0.44,0.63}{{#1}}}
    \newcommand{\StringTok}[1]{\textcolor[rgb]{0.25,0.44,0.63}{{#1}}}
    \newcommand{\CommentTok}[1]{\textcolor[rgb]{0.38,0.63,0.69}{\textit{{#1}}}}
    \newcommand{\OtherTok}[1]{\textcolor[rgb]{0.00,0.44,0.13}{{#1}}}
    \newcommand{\AlertTok}[1]{\textcolor[rgb]{1.00,0.00,0.00}{\textbf{{#1}}}}
    \newcommand{\FunctionTok}[1]{\textcolor[rgb]{0.02,0.16,0.49}{{#1}}}
    \newcommand{\RegionMarkerTok}[1]{{#1}}
    \newcommand{\ErrorTok}[1]{\textcolor[rgb]{1.00,0.00,0.00}{\textbf{{#1}}}}
    \newcommand{\NormalTok}[1]{{#1}}
    
    % Define a nice break command that doesn't care if a line doesn't already
    % exist.
    \def\br{\hspace*{\fill} \\* }
    % Math Jax compatability definitions
    \def\gt{>}
    \def\lt{<}
    % Document parameters
    \title{Introducao}
    
    
    

    % Pygments definitions
    
\makeatletter
\def\PY@reset{\let\PY@it=\relax \let\PY@bf=\relax%
    \let\PY@ul=\relax \let\PY@tc=\relax%
    \let\PY@bc=\relax \let\PY@ff=\relax}
\def\PY@tok#1{\csname PY@tok@#1\endcsname}
\def\PY@toks#1+{\ifx\relax#1\empty\else%
    \PY@tok{#1}\expandafter\PY@toks\fi}
\def\PY@do#1{\PY@bc{\PY@tc{\PY@ul{%
    \PY@it{\PY@bf{\PY@ff{#1}}}}}}}
\def\PY#1#2{\PY@reset\PY@toks#1+\relax+\PY@do{#2}}

\expandafter\def\csname PY@tok@gd\endcsname{\def\PY@tc##1{\textcolor[rgb]{0.63,0.00,0.00}{##1}}}
\expandafter\def\csname PY@tok@gu\endcsname{\let\PY@bf=\textbf\def\PY@tc##1{\textcolor[rgb]{0.50,0.00,0.50}{##1}}}
\expandafter\def\csname PY@tok@gt\endcsname{\def\PY@tc##1{\textcolor[rgb]{0.00,0.27,0.87}{##1}}}
\expandafter\def\csname PY@tok@gs\endcsname{\let\PY@bf=\textbf}
\expandafter\def\csname PY@tok@gr\endcsname{\def\PY@tc##1{\textcolor[rgb]{1.00,0.00,0.00}{##1}}}
\expandafter\def\csname PY@tok@cm\endcsname{\let\PY@it=\textit\def\PY@tc##1{\textcolor[rgb]{0.25,0.50,0.50}{##1}}}
\expandafter\def\csname PY@tok@vg\endcsname{\def\PY@tc##1{\textcolor[rgb]{0.10,0.09,0.49}{##1}}}
\expandafter\def\csname PY@tok@m\endcsname{\def\PY@tc##1{\textcolor[rgb]{0.40,0.40,0.40}{##1}}}
\expandafter\def\csname PY@tok@mh\endcsname{\def\PY@tc##1{\textcolor[rgb]{0.40,0.40,0.40}{##1}}}
\expandafter\def\csname PY@tok@go\endcsname{\def\PY@tc##1{\textcolor[rgb]{0.53,0.53,0.53}{##1}}}
\expandafter\def\csname PY@tok@ge\endcsname{\let\PY@it=\textit}
\expandafter\def\csname PY@tok@vc\endcsname{\def\PY@tc##1{\textcolor[rgb]{0.10,0.09,0.49}{##1}}}
\expandafter\def\csname PY@tok@il\endcsname{\def\PY@tc##1{\textcolor[rgb]{0.40,0.40,0.40}{##1}}}
\expandafter\def\csname PY@tok@cs\endcsname{\let\PY@it=\textit\def\PY@tc##1{\textcolor[rgb]{0.25,0.50,0.50}{##1}}}
\expandafter\def\csname PY@tok@cp\endcsname{\def\PY@tc##1{\textcolor[rgb]{0.74,0.48,0.00}{##1}}}
\expandafter\def\csname PY@tok@gi\endcsname{\def\PY@tc##1{\textcolor[rgb]{0.00,0.63,0.00}{##1}}}
\expandafter\def\csname PY@tok@gh\endcsname{\let\PY@bf=\textbf\def\PY@tc##1{\textcolor[rgb]{0.00,0.00,0.50}{##1}}}
\expandafter\def\csname PY@tok@ni\endcsname{\let\PY@bf=\textbf\def\PY@tc##1{\textcolor[rgb]{0.60,0.60,0.60}{##1}}}
\expandafter\def\csname PY@tok@nl\endcsname{\def\PY@tc##1{\textcolor[rgb]{0.63,0.63,0.00}{##1}}}
\expandafter\def\csname PY@tok@nn\endcsname{\let\PY@bf=\textbf\def\PY@tc##1{\textcolor[rgb]{0.00,0.00,1.00}{##1}}}
\expandafter\def\csname PY@tok@no\endcsname{\def\PY@tc##1{\textcolor[rgb]{0.53,0.00,0.00}{##1}}}
\expandafter\def\csname PY@tok@na\endcsname{\def\PY@tc##1{\textcolor[rgb]{0.49,0.56,0.16}{##1}}}
\expandafter\def\csname PY@tok@nb\endcsname{\def\PY@tc##1{\textcolor[rgb]{0.00,0.50,0.00}{##1}}}
\expandafter\def\csname PY@tok@nc\endcsname{\let\PY@bf=\textbf\def\PY@tc##1{\textcolor[rgb]{0.00,0.00,1.00}{##1}}}
\expandafter\def\csname PY@tok@nd\endcsname{\def\PY@tc##1{\textcolor[rgb]{0.67,0.13,1.00}{##1}}}
\expandafter\def\csname PY@tok@ne\endcsname{\let\PY@bf=\textbf\def\PY@tc##1{\textcolor[rgb]{0.82,0.25,0.23}{##1}}}
\expandafter\def\csname PY@tok@nf\endcsname{\def\PY@tc##1{\textcolor[rgb]{0.00,0.00,1.00}{##1}}}
\expandafter\def\csname PY@tok@si\endcsname{\let\PY@bf=\textbf\def\PY@tc##1{\textcolor[rgb]{0.73,0.40,0.53}{##1}}}
\expandafter\def\csname PY@tok@s2\endcsname{\def\PY@tc##1{\textcolor[rgb]{0.73,0.13,0.13}{##1}}}
\expandafter\def\csname PY@tok@vi\endcsname{\def\PY@tc##1{\textcolor[rgb]{0.10,0.09,0.49}{##1}}}
\expandafter\def\csname PY@tok@nt\endcsname{\let\PY@bf=\textbf\def\PY@tc##1{\textcolor[rgb]{0.00,0.50,0.00}{##1}}}
\expandafter\def\csname PY@tok@nv\endcsname{\def\PY@tc##1{\textcolor[rgb]{0.10,0.09,0.49}{##1}}}
\expandafter\def\csname PY@tok@s1\endcsname{\def\PY@tc##1{\textcolor[rgb]{0.73,0.13,0.13}{##1}}}
\expandafter\def\csname PY@tok@kd\endcsname{\let\PY@bf=\textbf\def\PY@tc##1{\textcolor[rgb]{0.00,0.50,0.00}{##1}}}
\expandafter\def\csname PY@tok@sh\endcsname{\def\PY@tc##1{\textcolor[rgb]{0.73,0.13,0.13}{##1}}}
\expandafter\def\csname PY@tok@sc\endcsname{\def\PY@tc##1{\textcolor[rgb]{0.73,0.13,0.13}{##1}}}
\expandafter\def\csname PY@tok@sx\endcsname{\def\PY@tc##1{\textcolor[rgb]{0.00,0.50,0.00}{##1}}}
\expandafter\def\csname PY@tok@bp\endcsname{\def\PY@tc##1{\textcolor[rgb]{0.00,0.50,0.00}{##1}}}
\expandafter\def\csname PY@tok@c1\endcsname{\let\PY@it=\textit\def\PY@tc##1{\textcolor[rgb]{0.25,0.50,0.50}{##1}}}
\expandafter\def\csname PY@tok@kc\endcsname{\let\PY@bf=\textbf\def\PY@tc##1{\textcolor[rgb]{0.00,0.50,0.00}{##1}}}
\expandafter\def\csname PY@tok@c\endcsname{\let\PY@it=\textit\def\PY@tc##1{\textcolor[rgb]{0.25,0.50,0.50}{##1}}}
\expandafter\def\csname PY@tok@mf\endcsname{\def\PY@tc##1{\textcolor[rgb]{0.40,0.40,0.40}{##1}}}
\expandafter\def\csname PY@tok@err\endcsname{\def\PY@bc##1{\setlength{\fboxsep}{0pt}\fcolorbox[rgb]{1.00,0.00,0.00}{1,1,1}{\strut ##1}}}
\expandafter\def\csname PY@tok@mb\endcsname{\def\PY@tc##1{\textcolor[rgb]{0.40,0.40,0.40}{##1}}}
\expandafter\def\csname PY@tok@ss\endcsname{\def\PY@tc##1{\textcolor[rgb]{0.10,0.09,0.49}{##1}}}
\expandafter\def\csname PY@tok@sr\endcsname{\def\PY@tc##1{\textcolor[rgb]{0.73,0.40,0.53}{##1}}}
\expandafter\def\csname PY@tok@mo\endcsname{\def\PY@tc##1{\textcolor[rgb]{0.40,0.40,0.40}{##1}}}
\expandafter\def\csname PY@tok@kn\endcsname{\let\PY@bf=\textbf\def\PY@tc##1{\textcolor[rgb]{0.00,0.50,0.00}{##1}}}
\expandafter\def\csname PY@tok@mi\endcsname{\def\PY@tc##1{\textcolor[rgb]{0.40,0.40,0.40}{##1}}}
\expandafter\def\csname PY@tok@gp\endcsname{\let\PY@bf=\textbf\def\PY@tc##1{\textcolor[rgb]{0.00,0.00,0.50}{##1}}}
\expandafter\def\csname PY@tok@o\endcsname{\def\PY@tc##1{\textcolor[rgb]{0.40,0.40,0.40}{##1}}}
\expandafter\def\csname PY@tok@kr\endcsname{\let\PY@bf=\textbf\def\PY@tc##1{\textcolor[rgb]{0.00,0.50,0.00}{##1}}}
\expandafter\def\csname PY@tok@s\endcsname{\def\PY@tc##1{\textcolor[rgb]{0.73,0.13,0.13}{##1}}}
\expandafter\def\csname PY@tok@kp\endcsname{\def\PY@tc##1{\textcolor[rgb]{0.00,0.50,0.00}{##1}}}
\expandafter\def\csname PY@tok@w\endcsname{\def\PY@tc##1{\textcolor[rgb]{0.73,0.73,0.73}{##1}}}
\expandafter\def\csname PY@tok@kt\endcsname{\def\PY@tc##1{\textcolor[rgb]{0.69,0.00,0.25}{##1}}}
\expandafter\def\csname PY@tok@ow\endcsname{\let\PY@bf=\textbf\def\PY@tc##1{\textcolor[rgb]{0.67,0.13,1.00}{##1}}}
\expandafter\def\csname PY@tok@sb\endcsname{\def\PY@tc##1{\textcolor[rgb]{0.73,0.13,0.13}{##1}}}
\expandafter\def\csname PY@tok@k\endcsname{\let\PY@bf=\textbf\def\PY@tc##1{\textcolor[rgb]{0.00,0.50,0.00}{##1}}}
\expandafter\def\csname PY@tok@se\endcsname{\let\PY@bf=\textbf\def\PY@tc##1{\textcolor[rgb]{0.73,0.40,0.13}{##1}}}
\expandafter\def\csname PY@tok@sd\endcsname{\let\PY@it=\textit\def\PY@tc##1{\textcolor[rgb]{0.73,0.13,0.13}{##1}}}

\def\PYZbs{\char`\\}
\def\PYZus{\char`\_}
\def\PYZob{\char`\{}
\def\PYZcb{\char`\}}
\def\PYZca{\char`\^}
\def\PYZam{\char`\&}
\def\PYZlt{\char`\<}
\def\PYZgt{\char`\>}
\def\PYZsh{\char`\#}
\def\PYZpc{\char`\%}
\def\PYZdl{\char`\$}
\def\PYZhy{\char`\-}
\def\PYZsq{\char`\'}
\def\PYZdq{\char`\"}
\def\PYZti{\char`\~}
% for compatibility with earlier versions
\def\PYZat{@}
\def\PYZlb{[}
\def\PYZrb{]}
\makeatother


    % Exact colors from NB
    \definecolor{incolor}{rgb}{0.0, 0.0, 0.5}
    \definecolor{outcolor}{rgb}{0.545, 0.0, 0.0}



    
    % Prevent overflowing lines due to hard-to-break entities
    \sloppy 
    % Setup hyperref package
    \hypersetup{
      breaklinks=true,  % so long urls are correctly broken across lines
      colorlinks=true,
      urlcolor=blue,
      linkcolor=darkorange,
      citecolor=darkgreen,
      }
    % Slightly bigger margins than the latex defaults
    
    \geometry{verbose,tmargin=1in,bmargin=1in,lmargin=1in,rmargin=1in}
    
    

    \begin{document}
    
    
    \maketitle
    
    

    

    \section{Introdução}


    \begin{Verbatim}[commandchars=\\\{\}]
{\color{incolor}In [{\color{incolor}2}]:} \PY{o}{\PYZpc{}}\PY{k}{matplotlib} \PY{n}{inline}
\end{Verbatim}

    \subsection{Matemática Discreta}\label{matemuxe1tica-discreta}

As Diretrizes Curriculares do MEC para os cursos de computação e
informática definem que:

\begin{quote}
A matemática, para a área de computação, deve ser vista como uma
ferramenta a ser usada na definição formal de conceitos computacionais
(linguagens, autômatos, métodos etc.). Os modelos formais permitem
definir suas propriedades e dimensionar suas instâncias, dadas suas
condições de contorno.
\end{quote}

Além disso, afirma:

\begin{quote}
Considerando que a maioria dos conceitos computacionais pertencem ao
domínio discreto, a \textbf{matemática discreta} (ou também chamada
álgebra abstrata) é fortemente empregada.
\end{quote}

Desta forma, a \textbf{Matemática Discreta} preocupa-se com o emprego de
técnicas e abordagens da matemática para o entendimento de problemas a
serem resolvidos com computação. Mas o que significa ser
\textbf{discreto}? A matemática, por si, trata também do domínio
\textbf{contínuo}. Assim, estes domínios são opostos: contínuo e
discreto. Para entender isso melhor, observe o gráfico a seguir:

    \begin{Verbatim}[commandchars=\\\{\}]
{\color{incolor}In [{\color{incolor}3}]:} \PY{k+kn}{from} \PY{n+nn}{pylab} \PY{k+kn}{import} \PY{o}{*}
\end{Verbatim}

    \begin{Verbatim}[commandchars=\\\{\}]
{\color{incolor}In [{\color{incolor}10}]:} \PY{n}{x} \PY{o}{=} \PY{n}{linspace}\PY{p}{(}\PY{l+m+mi}{0}\PY{p}{,} \PY{l+m+mi}{5}\PY{p}{,} \PY{l+m+mi}{5}\PY{p}{)}
         \PY{n}{y} \PY{o}{=} \PY{n}{x} \PY{o}{*}\PY{o}{*} \PY{l+m+mi}{2}
         \PY{n}{subplot}\PY{p}{(}\PY{l+m+mi}{1}\PY{p}{,}\PY{l+m+mi}{2}\PY{p}{,}\PY{l+m+mi}{1}\PY{p}{)}
         \PY{n}{plot}\PY{p}{(}\PY{n}{x}\PY{p}{,} \PY{n}{y}\PY{p}{,} \PY{l+s}{\PYZsq{}}\PY{l+s}{r\PYZhy{}}\PY{l+s}{\PYZsq{}}\PY{p}{)}
         \PY{n}{subplot}\PY{p}{(}\PY{l+m+mi}{1}\PY{p}{,}\PY{l+m+mi}{2}\PY{p}{,}\PY{l+m+mi}{2}\PY{p}{)}
         \PY{n}{plot}\PY{p}{(}\PY{n}{x}\PY{p}{,} \PY{n}{y}\PY{p}{,} \PY{l+s}{\PYZsq{}}\PY{l+s}{g*\PYZhy{}}\PY{l+s}{\PYZsq{}}\PY{p}{)}\PY{p}{;}
\end{Verbatim}

    \begin{center}
    \adjustimage{max size={0.9\linewidth}{0.9\paperheight}}{Introducao_files/Introducao_4_0.png}
    \end{center}
    { \hspace*{\fill} \\}
    
    O gráfico representa a função \(y = x^2\), com \(0 \leq x \leq 5\). O
gráfico da direita destaca os pontos selecionados. Neste gráfico, há 6
amostras, de 0 a 6.

Aumentando-se o número de amostras em dois instantes, para 100 e para
1000 tem-se:

    \begin{Verbatim}[commandchars=\\\{\}]
{\color{incolor}In [{\color{incolor}11}]:} \PY{n}{x1} \PY{o}{=} \PY{n}{linspace}\PY{p}{(}\PY{l+m+mi}{0}\PY{p}{,} \PY{l+m+mi}{5}\PY{p}{,} \PY{l+m+mi}{100}\PY{p}{)}
         \PY{n}{y1} \PY{o}{=} \PY{n}{x1} \PY{o}{*}\PY{o}{*} \PY{l+m+mi}{2}
         \PY{n}{x2} \PY{o}{=} \PY{n}{linspace}\PY{p}{(}\PY{l+m+mi}{0}\PY{p}{,} \PY{l+m+mi}{5}\PY{p}{,} \PY{l+m+mi}{1000}\PY{p}{)}
         \PY{n}{y2} \PY{o}{=} \PY{n}{x2} \PY{o}{*}\PY{o}{*} \PY{l+m+mi}{2}
         \PY{n}{subplot}\PY{p}{(}\PY{l+m+mi}{1}\PY{p}{,}\PY{l+m+mi}{2}\PY{p}{,}\PY{l+m+mi}{1}\PY{p}{)}
         \PY{n}{plot}\PY{p}{(}\PY{n}{x1}\PY{p}{,} \PY{n}{y1}\PY{p}{,} \PY{l+s}{\PYZsq{}}\PY{l+s}{r\PYZhy{}}\PY{l+s}{\PYZsq{}}\PY{p}{)}
         \PY{n}{subplot}\PY{p}{(}\PY{l+m+mi}{1}\PY{p}{,}\PY{l+m+mi}{2}\PY{p}{,}\PY{l+m+mi}{2}\PY{p}{)}
         \PY{n}{plot}\PY{p}{(}\PY{n}{x2}\PY{p}{,} \PY{n}{y2}\PY{p}{,} \PY{l+s}{\PYZsq{}}\PY{l+s}{r\PYZhy{}}\PY{l+s}{\PYZsq{}}\PY{p}{)}\PY{p}{;}
\end{Verbatim}

    \begin{center}
    \adjustimage{max size={0.9\linewidth}{0.9\paperheight}}{Introducao_files/Introducao_6_0.png}
    \end{center}
    { \hspace*{\fill} \\}
    
    O que se pode perceber é que quanto mais se aumenta o número de
amostras, mais se aproxima de uma curva perfeita. Entretanto, há um
certo limite de percepção da perfeição dessa curva, por assim dizer. Por
exemplo, embora a quantidade de amostras do gráfico da esquerda seja
menor, a diferença para o gráfico da direita, visualmente falando, é
pouco perceptível.

Considere outro exemplo: um computador possui uma capacidade de
armazenamento virtualmente infinita. ``Virtualmente'' porque embora se
aceite um limite, ele não é conhecido, já que a quantidade de unidades
de armazemamento pode ser bastante grande. Assim, no contexto da
computação, embora algo possa ser considerado finito ou infinito, ele é
\emph{contável} ou \emph{discreto} no sentido de que pode ser enumerado
ou sequenciado, de forma que não existe um elemento entre quaisquer dois
elementos consecutivos da enumeração. Em outras palavras, se um
computador possui dois discos rígidos (D1 e D2), não surgirá, do nada,
um terceiro disco ou meio disco entre D1 e D2.

No exemplo do computador, embora a quantidade de unidades de
armazenamento não seja conhecida, ela é contável e enumerável. Na
matemática, o conjunto dos números naturais é contável, equanto o
conjunto dos números reais não é contável.

Assim, a matemática discreta possui como ênfase os estudos matemáticos
baseados em conjuntos contáveis, sejam eles finitos ou infinitos. De
forma oposta, a \emph{matemática do continuum} possui ênfase nos
conjuntos não contáveis. Um exemplo disso são o cálculo diferencial e
integral.


    \subsection{Teoria dos conjuntos}


    Os \textbf{conjuntos} são a base da forma de representação de
enumerações de elementos em matemática discreta. Por definição um
conjunto é:

\begin{quote}
uma estrutura que agrupa objetos e constitui uma base para construir
estruturas mais complexas.
\end{quote}

Em outras palavras, um conjunto é uma coleção, ou uma lista, de
elementos.

Segue uma definição mais formal:

\begin{quote}
Um \emph{conjunto} é uma coleção de zero ou mais objetos distintos,
chamados \emph{elementos} do conjunto, os quais não possuem qualquer
ordem associada.
\end{quote}

O fato de não haver uma \emph{ordem associada} não significa que os
elementos não possam estar ordenados, num dado contexto, conforme algum
critério. Apenas indica que, no geral, isso não é obrigatório.

Há três formas de representar conjuntos: notação por extensão e notação
por compreensão.

\emph{Notação por extensão} é quando todos os elementos do conjunto
estão enumerados, representados entre chaves e separados por vírgula.
Exemplo:

\(\mbox{Vogais} = \{a, e, i, o, u\}\).

Entende-se que se um conjunto pode ser representado por extensão, então
ele é \emph{finito}. Caso contrário, é \emph{infinito}.

\emph{Notação por compreensão}: quando é usada uma representação por
propriedades. Os exemplos usam uma pequena diferença de notação, mas
representam a mesma coisa:

\begin{itemize}
\itemsep1pt\parskip0pt\parsep0pt
\item
  \(\mbox{Pares} = \{ n | n \mbox{ é um número par}\}\)
\item
  \(\mbox{Pares} = \{ n : n \mbox{ é um número par}\}\)
\end{itemize}

Este conjunto é interpretado como: o conjunto de todos os elementos
\(n\) tal que \(n\) é um número par. A forma geral de representar um
conjunto por propriedades é:

\(X = \{x : p(x)\}\)

Isso quer dizer que \(x\) é um elemento de \(X\) se a propriedade
\(p(x)\) for verdadeira.

A notação por propriedades é uma boa forma de representar conjuntos
\emph{infinitos}.

Há ainda uma outra forma aceitável de representar conjuntos usando uma
representação semelhante à de por extensão. Exemplos: -
\(\mbox{Digitos} = \{0, 1, 2, ..., 9\}\) -
\(\mbox{Pares} = \{0, 2, 4, 6, ...\}\)

Embora haja elementos ausentes, substituídos por reticências (\(...\)) é
completamente aceitável e entendível o que se quer informar com a
descrição do conjunto.

A seguir, revemos conceitos de algumas relações entre e com conjuntos ou
elementos.


    \subsubsection{Pertinência}


    Se um elemento \(a\) pertence ao conjunto \(A\) isso é representado
como: \(a \in A\). Caso contrário, se \(a\) não pertence a \(A\), então
representa-se como: \(a \not\in A\).

\textbf{Exemplos}: Pertence, não pertence: - Quanto ao conjunto
\(\mbox{Vogais} = \{a, e, i, o, u\}\): - \(a \in \mbox{Vogais}\) -
\(h \not\in \mbox{Vogais}\) - Quanto ao conjunto
\(B = \{x : x \mbox{ é brasileiro}\}\): - \(\mbox{Pele} \in B\) -
\(\mbox{Bill Gates} \not\in B\)


    \subsubsection{Conjuntos importantes}


    O \textbf{conjunto vazio} é um conjunto sem elementos, representado como
\(\{\}\) ou \(\emptyset\). Exemplos: - o conjunto de todos os
brasileiros com mais de 300 anos; - o conjunto dos números que são,
simultaneamente, ímpares e pares.

O \textbf{conjunto unitário} é um conjunto constituído por um único
elemento. Exemplos: - o conjunto constituído pelo jogador de futebol
Pelé; - o conjunto de todos os números que são, simultaneamente, pares e
primos, ou seja: \(P = \{2\}\); - um conjunto unitário cujo elemento é
irrelevante: \(1 = \{*\}\).

O \textbf{conjunto universo}, normalmente denotado por \(U\), contém
todos os conjuntos considerados em um dado contexto. Por isso, não é
fixo (pois depende do contexto).

Outros conjuntos importantes: - \(N\): o conjunto dos números naturais
(inteiros positivos e o zero) - \(Z\): o conjunto dos números inteiros
(inteiros negativos, positivos e o zero) - \(Q\): o conjunto dos números
racionais (os que podem ser representados na forma de fração) - \(I\): o
conjunto dos números irracionais - \(R\): o conjunto dos números reais


    \subsubsection{Alfabetos, palavras e linguagens}


    Em computação, e mais especificamente em linguagens de programação, um
conceito importante é o que define o conjunto de elementos ou
termos-chave da linguagem.

Um \textbf{alfabeto} é:

\begin{quote}
um conjunto finito cujos elementos são denominados \emph{símbolos} ou
\emph{caracteres}.
\end{quote}

Uma \textbf{palavra} (cadeia de caracteres ou sentença) sobre um
alfabeto é:

\begin{quote}
uma sequência finita de símbolos justapostos.
\end{quote}

Uma \textbf{linguagem {[}formal{]}} é

\begin{quote}
um conjunto de palavras sobre um alfabeto.
\end{quote}

\textbf{Exemplos}: alfabeto, palavra - Os conjuntos \(\emptyset\) e
\(\{a, b, c\}\) são alfabetos - O conjunto \(N\) não é um alfabeto -
\(\epsilon\) é uma palavra vazia - \(\Sigma\) é geralmente usada para
representar um alfabeto - \(\Sigma^*\) é o conjunto de todas as palavras
possíveis sobre o alfabeto \(\Sigma\) - \(\epsilon\) é uma palavra do
alfabeto \(\emptyset\) -
\(\{a, b\}^* = \{\epsilon, a, b, aa, ab, ba, bb, aaa, ...\}\)


    \subsubsection{Subconjunto e igualdade de conjuntos}


    A \emph{continência} permite introduzir os conceitos de
\emph{subconjunto} e \emph{igualdade de conjunto}.

Se todos os elementos de um conjunto \(A\) também são elementos de um
conjunto \(B\), então \(A\) está \emph{contido} em \(B\), o que é
representado por: \(A \subseteq B\). Isso também é lido como \(A\) é
\emph{subconjunto} de \(B\).

Se \(A \subseteq B\), mas há \(b \in B\) tal que \(b \not\in A\), então
pode-se dizer que \(A\) está \emph{contido propriamente} em \(B\), ou
que \(A\) é \emph{subconjunto próprio} de \(B\). Isso é denotado por:
\(A \subset B\).

A negação de \emph{subconjunto} e \emph{subconjunto próprio} é,
respectivamente: - \(A \not\subseteq B\) e - \(A \not\subset B\)

\textbf{Exemplos}: continência, subconjunto -
\(\{a, b\} \subseteq \{b, a\}\) - \(\{a, b\} \subset \{a, b, c\}\), e
\(\{a, b\} \subseteq \{a, b, c\}\)

Se os elementos de \(A\) também são elementos de \(B\) e vice-versa,
então \(A = B\). Formalmente, uma condição para \(A = B\) é que
\(A \subseteq B\) e \(B \subseteq A\).

\textbf{Exemplo}: - \(\{1, 2, 3\} = \{3, 3, 3, 2, 2, 1\}\)

É importante notar que pertinência (\(\in\)) é usado entre elementos e
conjuntos, enquanto continência (\(\subset\) e \(\subseteq\)) é usada
entre conjuntos.

Por definição, um conjunto qualquer é subconjunto de si mesmo, e
\(\emptyset\) é subconjunto de qualquer conjunto.

\textbf{Exemplo}: - Seja \(A = \{1, 2\}\) então os subconjuntos de \(A\)
são: \(\emptyset\), \(\{1\}\), \(\{2\}\) e \(\{1, 2\}\).


    % Add a bibliography block to the postdoc
    
    
    
    \end{document}
