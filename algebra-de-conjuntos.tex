
% Default to the notebook output style

    


% Inherit from the specified cell style.




    
\documentclass{article}

    
    
    \usepackage{graphicx} % Used to insert images
    \usepackage{adjustbox} % Used to constrain images to a maximum size 
    \usepackage{color} % Allow colors to be defined
    \usepackage{enumerate} % Needed for markdown enumerations to work
    \usepackage{geometry} % Used to adjust the document margins
    \usepackage{amsmath} % Equations
    \usepackage{amssymb} % Equations
    \usepackage[mathletters]{ucs} % Extended unicode (utf-8) support
    \usepackage[utf8x]{inputenc} % Allow utf-8 characters in the tex document
    \usepackage{fancyvrb} % verbatim replacement that allows latex
    \usepackage{grffile} % extends the file name processing of package graphics 
                         % to support a larger range 
    % The hyperref package gives us a pdf with properly built
    % internal navigation ('pdf bookmarks' for the table of contents,
    % internal cross-reference links, web links for URLs, etc.)
    \usepackage{hyperref}
    \usepackage{longtable} % longtable support required by pandoc >1.10
    \usepackage{booktabs}  % table support for pandoc > 1.12.2
    

    
    
    \definecolor{orange}{cmyk}{0,0.4,0.8,0.2}
    \definecolor{darkorange}{rgb}{.71,0.21,0.01}
    \definecolor{darkgreen}{rgb}{.12,.54,.11}
    \definecolor{myteal}{rgb}{.26, .44, .56}
    \definecolor{gray}{gray}{0.45}
    \definecolor{lightgray}{gray}{.95}
    \definecolor{mediumgray}{gray}{.8}
    \definecolor{inputbackground}{rgb}{.95, .95, .85}
    \definecolor{outputbackground}{rgb}{.95, .95, .95}
    \definecolor{traceback}{rgb}{1, .95, .95}
    % ansi colors
    \definecolor{red}{rgb}{.6,0,0}
    \definecolor{green}{rgb}{0,.65,0}
    \definecolor{brown}{rgb}{0.6,0.6,0}
    \definecolor{blue}{rgb}{0,.145,.698}
    \definecolor{purple}{rgb}{.698,.145,.698}
    \definecolor{cyan}{rgb}{0,.698,.698}
    \definecolor{lightgray}{gray}{0.5}
    
    % bright ansi colors
    \definecolor{darkgray}{gray}{0.25}
    \definecolor{lightred}{rgb}{1.0,0.39,0.28}
    \definecolor{lightgreen}{rgb}{0.48,0.99,0.0}
    \definecolor{lightblue}{rgb}{0.53,0.81,0.92}
    \definecolor{lightpurple}{rgb}{0.87,0.63,0.87}
    \definecolor{lightcyan}{rgb}{0.5,1.0,0.83}
    
    % commands and environments needed by pandoc snippets
    % extracted from the output of `pandoc -s`
    \DefineVerbatimEnvironment{Highlighting}{Verbatim}{commandchars=\\\{\}}
    % Add ',fontsize=\small' for more characters per line
    \newenvironment{Shaded}{}{}
    \newcommand{\KeywordTok}[1]{\textcolor[rgb]{0.00,0.44,0.13}{\textbf{{#1}}}}
    \newcommand{\DataTypeTok}[1]{\textcolor[rgb]{0.56,0.13,0.00}{{#1}}}
    \newcommand{\DecValTok}[1]{\textcolor[rgb]{0.25,0.63,0.44}{{#1}}}
    \newcommand{\BaseNTok}[1]{\textcolor[rgb]{0.25,0.63,0.44}{{#1}}}
    \newcommand{\FloatTok}[1]{\textcolor[rgb]{0.25,0.63,0.44}{{#1}}}
    \newcommand{\CharTok}[1]{\textcolor[rgb]{0.25,0.44,0.63}{{#1}}}
    \newcommand{\StringTok}[1]{\textcolor[rgb]{0.25,0.44,0.63}{{#1}}}
    \newcommand{\CommentTok}[1]{\textcolor[rgb]{0.38,0.63,0.69}{\textit{{#1}}}}
    \newcommand{\OtherTok}[1]{\textcolor[rgb]{0.00,0.44,0.13}{{#1}}}
    \newcommand{\AlertTok}[1]{\textcolor[rgb]{1.00,0.00,0.00}{\textbf{{#1}}}}
    \newcommand{\FunctionTok}[1]{\textcolor[rgb]{0.02,0.16,0.49}{{#1}}}
    \newcommand{\RegionMarkerTok}[1]{{#1}}
    \newcommand{\ErrorTok}[1]{\textcolor[rgb]{1.00,0.00,0.00}{\textbf{{#1}}}}
    \newcommand{\NormalTok}[1]{{#1}}
    
    % Define a nice break command that doesn't care if a line doesn't already
    % exist.
    \def\br{\hspace*{\fill} \\* }
    % Math Jax compatability definitions
    \def\gt{>}
    \def\lt{<}
    % Document parameters
    \title{algebra-de-conjuntos}
    
    
    

    % Pygments definitions
    
\makeatletter
\def\PY@reset{\let\PY@it=\relax \let\PY@bf=\relax%
    \let\PY@ul=\relax \let\PY@tc=\relax%
    \let\PY@bc=\relax \let\PY@ff=\relax}
\def\PY@tok#1{\csname PY@tok@#1\endcsname}
\def\PY@toks#1+{\ifx\relax#1\empty\else%
    \PY@tok{#1}\expandafter\PY@toks\fi}
\def\PY@do#1{\PY@bc{\PY@tc{\PY@ul{%
    \PY@it{\PY@bf{\PY@ff{#1}}}}}}}
\def\PY#1#2{\PY@reset\PY@toks#1+\relax+\PY@do{#2}}

\expandafter\def\csname PY@tok@gd\endcsname{\def\PY@tc##1{\textcolor[rgb]{0.63,0.00,0.00}{##1}}}
\expandafter\def\csname PY@tok@gu\endcsname{\let\PY@bf=\textbf\def\PY@tc##1{\textcolor[rgb]{0.50,0.00,0.50}{##1}}}
\expandafter\def\csname PY@tok@gt\endcsname{\def\PY@tc##1{\textcolor[rgb]{0.00,0.27,0.87}{##1}}}
\expandafter\def\csname PY@tok@gs\endcsname{\let\PY@bf=\textbf}
\expandafter\def\csname PY@tok@gr\endcsname{\def\PY@tc##1{\textcolor[rgb]{1.00,0.00,0.00}{##1}}}
\expandafter\def\csname PY@tok@cm\endcsname{\let\PY@it=\textit\def\PY@tc##1{\textcolor[rgb]{0.25,0.50,0.50}{##1}}}
\expandafter\def\csname PY@tok@vg\endcsname{\def\PY@tc##1{\textcolor[rgb]{0.10,0.09,0.49}{##1}}}
\expandafter\def\csname PY@tok@m\endcsname{\def\PY@tc##1{\textcolor[rgb]{0.40,0.40,0.40}{##1}}}
\expandafter\def\csname PY@tok@mh\endcsname{\def\PY@tc##1{\textcolor[rgb]{0.40,0.40,0.40}{##1}}}
\expandafter\def\csname PY@tok@go\endcsname{\def\PY@tc##1{\textcolor[rgb]{0.53,0.53,0.53}{##1}}}
\expandafter\def\csname PY@tok@ge\endcsname{\let\PY@it=\textit}
\expandafter\def\csname PY@tok@vc\endcsname{\def\PY@tc##1{\textcolor[rgb]{0.10,0.09,0.49}{##1}}}
\expandafter\def\csname PY@tok@il\endcsname{\def\PY@tc##1{\textcolor[rgb]{0.40,0.40,0.40}{##1}}}
\expandafter\def\csname PY@tok@cs\endcsname{\let\PY@it=\textit\def\PY@tc##1{\textcolor[rgb]{0.25,0.50,0.50}{##1}}}
\expandafter\def\csname PY@tok@cp\endcsname{\def\PY@tc##1{\textcolor[rgb]{0.74,0.48,0.00}{##1}}}
\expandafter\def\csname PY@tok@gi\endcsname{\def\PY@tc##1{\textcolor[rgb]{0.00,0.63,0.00}{##1}}}
\expandafter\def\csname PY@tok@gh\endcsname{\let\PY@bf=\textbf\def\PY@tc##1{\textcolor[rgb]{0.00,0.00,0.50}{##1}}}
\expandafter\def\csname PY@tok@ni\endcsname{\let\PY@bf=\textbf\def\PY@tc##1{\textcolor[rgb]{0.60,0.60,0.60}{##1}}}
\expandafter\def\csname PY@tok@nl\endcsname{\def\PY@tc##1{\textcolor[rgb]{0.63,0.63,0.00}{##1}}}
\expandafter\def\csname PY@tok@nn\endcsname{\let\PY@bf=\textbf\def\PY@tc##1{\textcolor[rgb]{0.00,0.00,1.00}{##1}}}
\expandafter\def\csname PY@tok@no\endcsname{\def\PY@tc##1{\textcolor[rgb]{0.53,0.00,0.00}{##1}}}
\expandafter\def\csname PY@tok@na\endcsname{\def\PY@tc##1{\textcolor[rgb]{0.49,0.56,0.16}{##1}}}
\expandafter\def\csname PY@tok@nb\endcsname{\def\PY@tc##1{\textcolor[rgb]{0.00,0.50,0.00}{##1}}}
\expandafter\def\csname PY@tok@nc\endcsname{\let\PY@bf=\textbf\def\PY@tc##1{\textcolor[rgb]{0.00,0.00,1.00}{##1}}}
\expandafter\def\csname PY@tok@nd\endcsname{\def\PY@tc##1{\textcolor[rgb]{0.67,0.13,1.00}{##1}}}
\expandafter\def\csname PY@tok@ne\endcsname{\let\PY@bf=\textbf\def\PY@tc##1{\textcolor[rgb]{0.82,0.25,0.23}{##1}}}
\expandafter\def\csname PY@tok@nf\endcsname{\def\PY@tc##1{\textcolor[rgb]{0.00,0.00,1.00}{##1}}}
\expandafter\def\csname PY@tok@si\endcsname{\let\PY@bf=\textbf\def\PY@tc##1{\textcolor[rgb]{0.73,0.40,0.53}{##1}}}
\expandafter\def\csname PY@tok@s2\endcsname{\def\PY@tc##1{\textcolor[rgb]{0.73,0.13,0.13}{##1}}}
\expandafter\def\csname PY@tok@vi\endcsname{\def\PY@tc##1{\textcolor[rgb]{0.10,0.09,0.49}{##1}}}
\expandafter\def\csname PY@tok@nt\endcsname{\let\PY@bf=\textbf\def\PY@tc##1{\textcolor[rgb]{0.00,0.50,0.00}{##1}}}
\expandafter\def\csname PY@tok@nv\endcsname{\def\PY@tc##1{\textcolor[rgb]{0.10,0.09,0.49}{##1}}}
\expandafter\def\csname PY@tok@s1\endcsname{\def\PY@tc##1{\textcolor[rgb]{0.73,0.13,0.13}{##1}}}
\expandafter\def\csname PY@tok@kd\endcsname{\let\PY@bf=\textbf\def\PY@tc##1{\textcolor[rgb]{0.00,0.50,0.00}{##1}}}
\expandafter\def\csname PY@tok@sh\endcsname{\def\PY@tc##1{\textcolor[rgb]{0.73,0.13,0.13}{##1}}}
\expandafter\def\csname PY@tok@sc\endcsname{\def\PY@tc##1{\textcolor[rgb]{0.73,0.13,0.13}{##1}}}
\expandafter\def\csname PY@tok@sx\endcsname{\def\PY@tc##1{\textcolor[rgb]{0.00,0.50,0.00}{##1}}}
\expandafter\def\csname PY@tok@bp\endcsname{\def\PY@tc##1{\textcolor[rgb]{0.00,0.50,0.00}{##1}}}
\expandafter\def\csname PY@tok@c1\endcsname{\let\PY@it=\textit\def\PY@tc##1{\textcolor[rgb]{0.25,0.50,0.50}{##1}}}
\expandafter\def\csname PY@tok@kc\endcsname{\let\PY@bf=\textbf\def\PY@tc##1{\textcolor[rgb]{0.00,0.50,0.00}{##1}}}
\expandafter\def\csname PY@tok@c\endcsname{\let\PY@it=\textit\def\PY@tc##1{\textcolor[rgb]{0.25,0.50,0.50}{##1}}}
\expandafter\def\csname PY@tok@mf\endcsname{\def\PY@tc##1{\textcolor[rgb]{0.40,0.40,0.40}{##1}}}
\expandafter\def\csname PY@tok@err\endcsname{\def\PY@bc##1{\setlength{\fboxsep}{0pt}\fcolorbox[rgb]{1.00,0.00,0.00}{1,1,1}{\strut ##1}}}
\expandafter\def\csname PY@tok@mb\endcsname{\def\PY@tc##1{\textcolor[rgb]{0.40,0.40,0.40}{##1}}}
\expandafter\def\csname PY@tok@ss\endcsname{\def\PY@tc##1{\textcolor[rgb]{0.10,0.09,0.49}{##1}}}
\expandafter\def\csname PY@tok@sr\endcsname{\def\PY@tc##1{\textcolor[rgb]{0.73,0.40,0.53}{##1}}}
\expandafter\def\csname PY@tok@mo\endcsname{\def\PY@tc##1{\textcolor[rgb]{0.40,0.40,0.40}{##1}}}
\expandafter\def\csname PY@tok@kn\endcsname{\let\PY@bf=\textbf\def\PY@tc##1{\textcolor[rgb]{0.00,0.50,0.00}{##1}}}
\expandafter\def\csname PY@tok@mi\endcsname{\def\PY@tc##1{\textcolor[rgb]{0.40,0.40,0.40}{##1}}}
\expandafter\def\csname PY@tok@gp\endcsname{\let\PY@bf=\textbf\def\PY@tc##1{\textcolor[rgb]{0.00,0.00,0.50}{##1}}}
\expandafter\def\csname PY@tok@o\endcsname{\def\PY@tc##1{\textcolor[rgb]{0.40,0.40,0.40}{##1}}}
\expandafter\def\csname PY@tok@kr\endcsname{\let\PY@bf=\textbf\def\PY@tc##1{\textcolor[rgb]{0.00,0.50,0.00}{##1}}}
\expandafter\def\csname PY@tok@s\endcsname{\def\PY@tc##1{\textcolor[rgb]{0.73,0.13,0.13}{##1}}}
\expandafter\def\csname PY@tok@kp\endcsname{\def\PY@tc##1{\textcolor[rgb]{0.00,0.50,0.00}{##1}}}
\expandafter\def\csname PY@tok@w\endcsname{\def\PY@tc##1{\textcolor[rgb]{0.73,0.73,0.73}{##1}}}
\expandafter\def\csname PY@tok@kt\endcsname{\def\PY@tc##1{\textcolor[rgb]{0.69,0.00,0.25}{##1}}}
\expandafter\def\csname PY@tok@ow\endcsname{\let\PY@bf=\textbf\def\PY@tc##1{\textcolor[rgb]{0.67,0.13,1.00}{##1}}}
\expandafter\def\csname PY@tok@sb\endcsname{\def\PY@tc##1{\textcolor[rgb]{0.73,0.13,0.13}{##1}}}
\expandafter\def\csname PY@tok@k\endcsname{\let\PY@bf=\textbf\def\PY@tc##1{\textcolor[rgb]{0.00,0.50,0.00}{##1}}}
\expandafter\def\csname PY@tok@se\endcsname{\let\PY@bf=\textbf\def\PY@tc##1{\textcolor[rgb]{0.73,0.40,0.13}{##1}}}
\expandafter\def\csname PY@tok@sd\endcsname{\let\PY@it=\textit\def\PY@tc##1{\textcolor[rgb]{0.73,0.13,0.13}{##1}}}

\def\PYZbs{\char`\\}
\def\PYZus{\char`\_}
\def\PYZob{\char`\{}
\def\PYZcb{\char`\}}
\def\PYZca{\char`\^}
\def\PYZam{\char`\&}
\def\PYZlt{\char`\<}
\def\PYZgt{\char`\>}
\def\PYZsh{\char`\#}
\def\PYZpc{\char`\%}
\def\PYZdl{\char`\$}
\def\PYZhy{\char`\-}
\def\PYZsq{\char`\'}
\def\PYZdq{\char`\"}
\def\PYZti{\char`\~}
% for compatibility with earlier versions
\def\PYZat{@}
\def\PYZlb{[}
\def\PYZrb{]}
\makeatother


    % Exact colors from NB
    \definecolor{incolor}{rgb}{0.0, 0.0, 0.5}
    \definecolor{outcolor}{rgb}{0.545, 0.0, 0.0}



    
    % Prevent overflowing lines due to hard-to-break entities
    \sloppy 
    % Setup hyperref package
    \hypersetup{
      breaklinks=true,  % so long urls are correctly broken across lines
      colorlinks=true,
      urlcolor=blue,
      linkcolor=darkorange,
      citecolor=darkgreen,
      }
    % Slightly bigger margins than the latex defaults
    
    \geometry{verbose,tmargin=1in,bmargin=1in,lmargin=1in,rmargin=1in}
    
    

    \begin{document}
    
    
    \maketitle
    
    

    

    \section{Álgebra de Conjuntos}


    \begin{Verbatim}[commandchars=\\\{\}]
{\color{incolor}In [{\color{incolor}12}]:} \PY{o}{\PYZpc{}}\PY{k}{matplotlib} \PY{n}{inline}
\end{Verbatim}

    \begin{quote}
Álgebra é constituída de operações definidas sobre uma soleção de
objetos. Neste contexto, \emph{álgebra de conjuntos} corresponderia às
operações definidas sobre todos os conjuntos.
\end{quote}

As operações da álgebra de conjuntos podem ser \emph{não reversíveis} e
\emph{reversíveis}: - Não reversíveis - União - Intersecção - Diferença
- Reversíveis - Complemento - Conjunto das partes - Produto cartesiano -
União disjunta

É possível estabelecer uma relação entre a lógica e as operações da
álgebra de conjuntos:

\begin{longtable}[c]{@{}ll@{}}
\toprule
Conectivo ou relação lógicos & Operação ou relação sobre
conjuntos\tabularnewline
\midrule
\endhead
Negação & Complemento\tabularnewline
Disjunção & União\tabularnewline
Conjunção & Intersecção\tabularnewline
Implicação & Continência\tabularnewline
Equivalência & Igualdade\tabularnewline
\bottomrule
\end{longtable}

As propriedades dos conectivos e operadores lógicos são válidas na
teoria dos conjuntos, como mostrado a seguir:

\begin{longtable}[c]{@{}ll@{}}
\toprule
Conectivo lógico & Operação sobre conjuntos\tabularnewline
\midrule
\endhead
\textbf{Idempotência:} \(\wedge\) e \(\lor\) & idmpotência: intersecção
e união\tabularnewline
\textbf{Comutativa:} \(\wedge\) e \(\lor\) & comutativa: intersecção e
união\tabularnewline
\textbf{Associativa:} \(\wedge\) e \(\lor\) & associativa: intersecção e
união\tabularnewline
\textbf{Distributiva:}\(\wedge\) sobre \(\lor\)\(\lor\) sobre \(\wedge\)
& distributiva:intersecção sobre uniãounião sobre
intersecção\tabularnewline
\textbf{Dupla negação} & duplo complemento\tabularnewline
\textbf{DeMorgan} & DeMorgan\tabularnewline
\textbf{absorção} & absorção\tabularnewline
\bottomrule
\end{longtable}


    \subsection{Operações não reversíveis}



    \subsubsection{União}


    Sejam \(A\) e \(B\) dois conjuntos. A união entre eles, \(A \cup B\), é
definida como:

\(A \cup B = \{x | x \in A \lor x \in B\}\)

Considerando a lógica, o conjunto \(A\) pode ser definido como
\(x \in A\). O conjunto \(B\) pode ser definido como \(x \in B\). Ou
seja, a propriedade de pertiência é utilizada para indicar uma
proposição lógica.

A união corresponde à operação lógica \emph{disjunção} (símbolo
\(\lor\)).

\textbf{Exemplo:} Considere os conjuntos: -
\(\mbox{Digitos} = \{0, 1, 2, 3, 4, 5, 6, 7, 8, 9\}\) -
\(\mbox{Vogais} = \{a, e, i, o, u\}\) -
\(\mbox{Pares} = \{0, 2, 4, 6, ...\}\)

Então:

\begin{itemize}
\itemsep1pt\parskip0pt\parsep0pt
\item
  \(\mbox{Digitos} \cup \mbox{Vogais} = \{0, 1, 2, 3, 4, 5, 6, 7, 8, 9, a, e, i, o, u\}\)
\item
  \(\mbox{Digitos} \cup \mbox{Pares} = \{0, 1, 2, 3, 4, 5, 6, 7, 8, 9, 10, 12, 14, 16, ...\}\)
\end{itemize}

\textbf{Exemplo:} Suponha os conjuntos: - \(A = \{x \in N | x > 2\}\) -
\(B = \{x \in N | x^2 = x\}\)

Então: - \(A \cup B = \{0, 1, 3, 4, 5, 6, ...\}\)

\textbf{É importante} observar que o resultado da união é um conjunto
sem repetições de elementos.

Vejamos as propriedades da união: - \textbf{Elemento neutro}:
\(A \cup \emptyset = \emptyset \cup A = A\) - \textbf{Idempotência}:
\(A \cup A = A\) - \textbf{Comutativa}: \(A \cup B = B \cup A\) -
\textbf{Associativa}: \(A \cup (B \cup C) = (A \cup B) \cup C\)

\textbf{Prova da propriedade \emph{elemento neutro}} Elemento neutro:
\(A \cup \emptyset = \emptyset \cup A = A\).

Assim, há duas igualdades, que podem ser analisadas considerando a
validade da transitividade {[}da igualdade{]}. Assim, temos que observar
alguns casos.

\textbf{(A) Para provar \(A \cup \emptyset = \emptyset \cup A\)}:

\emph{O primeiro caso (1)}: Seja \(x \in A \cup \emptyset\). Então
devemos provar que \(A \cup \emptyset \subseteq \emptyset \cup A\): -
\(x \in A \cup \emptyset \implies\) (definição de união) -
\(x \in A \lor x \in \emptyset \implies\) (comutatividade da disjunção)
- \(x \in \emptyset \lor x \in A \implies\) (definição de união) -
\(x \in \emptyset \cup A\)

Portanto, \(A \cup \emptyset \subseteq \emptyset \cup A\).

\emph{O segundo caso (2)}: Seja \(x \in \emptyset \cup A\). Então
devemos provar que \(\emptyset \cup A \subseteq A \cup \emptyset\): -
\(x \in \emptyset \cup A \implies\) (definição de união) -
\(x \in \emptyset \lor x \in A \implies\) (comutatividade da disjunção)
- \(x \in A \lor x \in \emptyset \implies\) (definição de união) -
\(x \in A \cup \emptyset\)

Portanto, \(\emptyset \cup A = A \cup \emptyset\).

\emph{Terceiro caso (3)}: De (1) e (2) concluímos que
\(A \cup \emptyset = \emptyset \cup A\).

\textbf{(B) Para provar \(A \cup \emptyset = A\)}:

\emph{Quarto caso (4)}: Seja \(x \in A \cup \emptyset\). Então devemos
provar que \(A \cup \emptyset \subseteq A\): -
\(x \in A \cup \emptyset \implies\) (definição de união) -
\(x \in A \lor x \in \emptyset \implies\) (\(x \in \emptyset\) é sempre
\emph{false}) - \(x \in A\)

Portanto, \(A \cup \emptyset \subseteq A\).

\emph{Quinto caso (5)}: Seja \(x \in A\). Então devemos provar que
\(A \subseteq A \cup \emptyset\): - \(x \in A \implies\) (\(x \in A\) é
sempre \emph{true}, portanto podemos considerar \(p \implies p \lor q)\)
- \(x \in A \lor x \in \emptyset\) (definição de união) -
\(x \in A \cup \emptyset\)

Portanto, \(A \subseteq A \cup \emptyset\).

\emph{Sexto caso (6)}: De (4) e (5) concluímos que
\(A \cup \emptyset = A\).

\emph{Sétimo caso (7)}: Por fim, de (3) e (6) e pela transitividade da
igualdade, concluímos que \(A \cup \emptyset = \emptyset \cup A = A\) e
provamos a propriedade do \emph{elemento neutro} da união.

\textbf{Exercício 2.1}: Prove as propriedades idempotência, comutativa e
associativa da união.

    \begin{Verbatim}[commandchars=\\\{\}]
{\color{incolor}In [{\color{incolor}17}]:} \PY{k+kn}{import} \PY{n+nn}{pylab} \PY{k+kn}{as} \PY{n+nn}{plt}
         \PY{k+kn}{from} \PY{n+nn}{matplotlib\PYZus{}venn} \PY{k+kn}{import} \PY{n}{venn2}\PY{p}{,} \PY{n}{venn2\PYZus{}circles}
         \PY{n}{set1} \PY{o}{=} \PY{n+nb}{set}\PY{p}{(}\PY{p}{[}\PY{l+m+mi}{1}\PY{p}{,}\PY{l+m+mi}{2}\PY{p}{,}\PY{l+m+mi}{3}\PY{p}{]}\PY{p}{)}
         \PY{n}{set2} \PY{o}{=} \PY{n+nb}{set}\PY{p}{(}\PY{p}{[}\PY{l+m+mi}{3}\PY{p}{,}\PY{l+m+mi}{4}\PY{p}{,}\PY{l+m+mi}{5}\PY{p}{]}\PY{p}{)}
         \PY{n}{venn2}\PY{p}{(}\PY{p}{[}\PY{n}{set1}\PY{p}{,} \PY{n}{set2}\PY{p}{]}\PY{p}{,} \PY{p}{(}\PY{l+s}{\PYZsq{}}\PY{l+s}{A}\PY{l+s}{\PYZsq{}}\PY{p}{,} \PY{l+s}{\PYZsq{}}\PY{l+s}{B}\PY{l+s}{\PYZsq{}}\PY{p}{)}\PY{p}{)}
         \PY{n}{plt}\PY{o}{.}\PY{n}{show}\PY{p}{(}\PY{p}{)}
\end{Verbatim}

    \begin{center}
    \adjustimage{max size={0.9\linewidth}{0.9\paperheight}}{algebra-de-conjuntos_files/algebra-de-conjuntos_6_0.png}
    \end{center}
    { \hspace*{\fill} \\}
    

    \subsubsection{Intersecção}


    Sejam dois conjuntos \(A\) e \(B\). A intersercção entre eles,
\(A \cap B\) é definida como:

\(A \cap B = \{x | x \in A \wedge x \in B\}\)

A união corresponde à operação lógica \emph{conjunção} (símbolo
\(\wedge\)).

Sejam \(A\) e \(B\) dois conjuntos não vazios. Se
\(A \cap B = \emptyset\), então \(A\) e \(B\) são chamados
\emph{conjuntos disjuntos}, \emph{conjuntos independentes}, ou
\emph{conjuntos mutuamente exclusivos}.

\textbf{Exemplo:} Considere os conjuntos: -
\(\mbox{Digitos} = \{0, 1, 2, 3, 4, 5, 6, 7, 8, 9\}\) -
\(\mbox{Vogais} = \{a, e, i, o, u\}\) -
\(\mbox{Pares} = \{0, 2, 4, 6, ...\}\)

Então: - \(\mbox{Digitos} \cap \mbox{Vogais} = \emptyset\) -
\(\mbox{Digitos} \cap \mbox{Pares} = \{0, 2, 4, 6, 8\}\)

Vejamos as propriedades da intersecção: - \textbf{Elemento neutro}:
\(A \cap U = U \cap A = A\) - \textbf{Idempotência}: \(A \cap A = A\) -
\textbf{Comutativa}: \(A \cap B = B \cap A\) - \textbf{Associativa}:
\(A \cap (B \cap C) = (A \cap B) \cap C\)


    \subsubsection{Propriedades envolvendo união e intersecção}


    As propriedades a seguir envolvem as operações de união e intersecção: -
\textbf{Distributividade da intersecção sobre a união}:
\(A \cap (B \cup C) = (A \cap B) \cup (A \cap C)\) -
\textbf{Distributividade da união sobre a intersecção}:
\(A \cup (B \cap C) = (A \cup B) \cap (A \cup C)\) - \textbf{Absorção}:
\(A \cap (A \cup B) = A\) e \(A \cup (A \cap B) = A\).

    \textbf{Exercício 2.2}: Algumas linguagens de programação possuem
estruturas de dados para conjuntos, as quais disponibilizam, também,
operações sobre estes. Faça uma pesquisa sobre linguagens de programação
e, selecionando uma linguagem de programação que suporte definição de
conjuntos e operações sobre eles, apresente exemplos, contemplando as
operações e propriedades vistas até o momento (e.g.~pertiência,
contingência, união e intersecção).

\textbf{Exercício 2.3}: Considerando uma linguagem de programação que
forneça suporte a conjuntos e operações sobre eles, crie um programa que
leia conjuntos em arquivos texto (um elemento do conjunto em cada linha)
e gere a saída também em um arquivo texto (também um elemento em cada
linha). O programa deve considerar e demonstrar a utilização das
operações e propriedades vistas até o momento (e.g.~pertiência,
contingência, união e intersecção).

\textbf{Exercício 2.4}: Suponha o conjunto universo
\(S = \{p, q, r, s, t, u, v, w\}\) bem como os seguintes conjuntos: -
\(A = \{p, q, r, s\}\) - \(B = \{r, t, v\}\) - \(C = \{p, s, t, u\}\)

Determine:

\begin{enumerate}
\def\labelenumi{\alph{enumi})}
\item
  \(B \cap C\)
\item
  \(A \cup C\)
\item
  \(\sim{C}\)
\item
  \(A \cap B \cap C\)
\item
  \(\sim(A \cup B)\)
\item
  \((A \cup B) \cap \sim{C}\)
\end{enumerate}


    % Add a bibliography block to the postdoc
    
    
    
    \end{document}
